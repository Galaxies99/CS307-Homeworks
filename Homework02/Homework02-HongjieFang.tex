\documentclass[12pt,a4paper]{article}
\usepackage{ctex}
\usepackage{amsmath,amscd,amsbsy,amssymb,latexsym,url,bm,amsthm}
\usepackage{epsfig,graphicx,subfigure}
\usepackage{enumitem,balance}
\usepackage{enumerate}
\usepackage{wrapfig}
\usepackage{mathrsfs,euscript}
\usepackage[usenames]{xcolor}
\usepackage{hyperref}
\usepackage[vlined,ruled,linesnumbered]{algorithm2e}
\hypersetup{colorlinks=true,linkcolor=black}

\newtheorem{theorem}{Theorem}
\newtheorem{lemma}[theorem]{Lemma}
\newtheorem{proposition}[theorem]{Proposition}
\newtheorem{corollary}[theorem]{Corollary}
\newtheorem{exercise}{Exercise}
\newtheorem*{solution}{Solution}
\newtheorem{definition}{Definition}
\theoremstyle{definition}

\renewcommand{\thefootnote}{\fnsymbol{footnote}}

\newcommand{\postscript}[2]
 {\setlength{\epsfxsize}{#2\hsize}
  \centerline{\epsfbox{#1}}}

\renewcommand{\baselinestretch}{1.0}

\setlength{\oddsidemargin}{-0.365in}
\setlength{\evensidemargin}{-0.365in}
\setlength{\topmargin}{-0.3in}
\setlength{\headheight}{0in}
\setlength{\headsep}{0in}
\setlength{\textheight}{10.1in}
\setlength{\textwidth}{7in}
\makeatletter \renewenvironment{proof}[1][Proof] {\par\pushQED{\qed}\normalfont\topsep6\p@\@plus6\p@\relax\trivlist\item[\hskip\labelsep\bfseries#1\@addpunct{.}]\ignorespaces}{\popQED\endtrivlist\@endpefalse} \makeatother
\makeatletter
\renewenvironment{solution}[1][Solution] {\par\pushQED{\qed}\normalfont\topsep6\p@\@plus6\p@\relax\trivlist\item[\hskip\labelsep\bfseries#1\@addpunct{.}]\ignorespaces}{\popQED\endtrivlist\@endpefalse} \makeatother

\begin{document}
\noindent

%========================================================================
\noindent\framebox[\linewidth]{\shortstack[c]{
\Large{\textbf{Homework 02}}\vspace{1mm}\\
CS307-Operating System (D), Chentao Wu, Spring 2020.}}
\begin{center}
\footnotesize{\color{blue}Name: ������ (Hongjie Fang)  \quad Student ID: 518030910150 \quad Email: galaxies@sjtu.edu.cn}
\end{center}

\begin{itemize}
    \item (2.2) What is the purpose of the command interpreter? Why is it usually separate from the kernel?
    \begin{solution}
        The \textbf{purpose} of the command interpreter is to allow users to directly enter commands to be performed by the operating system. It reads commands from users or a file consists of commands, and executes them by transforming the commands to some system calls usually. 
        
        The command interpreter is an \textbf{interface} between users and operating system. To \textbf{protect the kernel} from some dangerous instructions which users may enter, the command interpreter should be separate from the kernel most of the time.
    \end{solution}
    
    \item (2.5) What is the main advantage of the layered approach to system design? What are the disadvantages of the layered approach?
    \begin{solution}
        The main \textbf{advantage} of the layered approach is \textbf{simplicity of construction and debugging}. We can simplify the process of constructing and debugging by following the order of the layer number. And modifications in a layer only affect a few other layers, so it's easy to make modifications.
        
        The \textbf{disadvantages} of the layered approach is as follows:
        \begin{itemize}
        \item It's hard to define the functionality of each layer appropriately.
        \item The overall performance of such systems is poor, because of the overhead of requiring a user program to traverse through multiple layers to obtain an operating-system service.
        \end{itemize}
    \end{solution}
    
    \item (2.7) Why do some systems store the operating system in firmware, while others store it on disk?
    \begin{solution}
        A disk along with a file system may \textbf{not} be \textbf{available} in some systems, so the operating system has to be stored in the firmware under this circumstance.
    \end{solution}
\end{itemize}
%========================================================================
\end{document}
